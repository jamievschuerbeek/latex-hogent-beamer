\documentclass[aspectratio=169]{beamer}

% Je kan het lettertype iets vergroten door hierboven optie ``14pt'' toe te
% voegen.

%==============================================================================
% Aanloop
%==============================================================================

%---------- Vormgeving --------------------------------------------------------

\usetheme{hogent}

% Kies hieronder een achtergrondkleur
\usecolortheme{hgwhite} % witte achtergrond, zwarte tekst
%\usecolortheme{hgblack} % zwarte achtergrond, witte tekst

%---------- Packages ----------------------------------------------------------

\usepackage[dutch]{babel}      % Nederlandse taal: splitsingen, enz.

\usepackage{booktabs}          % Mooie tabellen
\usepackage{multirow,multicol} % Tabelcellen samenvoegen
\usepackage{eurosym}           % Euro symbool

%---------- Commando-definities -----------------------------------------------

%---------- Info over de presentatie ------------------------------------------

\title{Machine learning-technologie toepassen om relevante masterdata uit bedrijfsdocumenten te halen}
\subtitle{Bachelorproef Toegepaste Informatica}
\author{Jamie Van Schuerbeek}
\date{\today}

%==============================================================================
% Inhoud presentatie
%==============================================================================

\begin{document}

%---------- Titelpagina, inhoudstafel -----------------------------------------

\frame{\maketitle}
 
%---------- Corpus -----------------------------------------------------------

\begin{frame}
  \frametitle{Probleemstelling.}
  
  
\end{frame}

\begin{frame}
  \frametitle{Onderzoeksvragen}
  \centering
  \textbf{Hoofdvraag}
\\
Hoe kan machine learning-technologie worden toegepast om relevante masterdata uit bedrijfsdocumenten te halen?
\\
\textbf{Deelvraag}
\\
Welke tools bestaan er die gebruikt kunnen worden een dergelijke oplossing te maken?
\end{frame}

\begin{frame}
  \frametitle{Werkwijze}

\end{frame}

\begin{frame}
  \frametitle{Resultaten}

\end{frame}

\begin{frame}
  \frametitle{Conclusie}

\end{frame}

\end{document}
